\section{Benutzerdefinierte Typen}

In Modulen werden häufig Datentypen vom Benutzer definiert, zum Beispiel der Datentyp \texttt{kreis}
\begin{lstlisting}
type kreis
 private !kein Zugriff auf Komponenten im Hauptprogramm
 real :: radius
 real :: mitteX
 real :: mitteY
end type kreis
\end{lstlisting}

Für diesen neuen Datentyp funktionieren die alten Operatoren wie \texttt{+} nicht mehr (was soll den die Summe aus 2 Kreisen sein?). Deswegen muss man sich eine neue Addition von Kreisen ausdenken:
\begin{lstlisting}
function add(kreis1, kreis2)
 type(kreis), intent(in) :: kreis1, kreis2
 type(kreis) :: add
 
 add%raduis = kreis1%radius + kreis2%radius
 add%mitteX = (kreis1%mitteX + kreis2%mitteX) / 2
 add%mitteY = (kreis1%mitteY + kreis2%mitteY) / 2
end function add
\end{lstlisting}

Wie man sieht kann man mit \% auf die einzelnen Komponenten zugreifen. Um jetzt wirklich im Programm \texttt{kreis1 + kreis2} zu benutzen, muss man noch den Operator \texttt{+} überladen. Dazu werden generische Schnittstellen benutzt. Man kann auch das Gleichheitszeichen, also die Zuweisung \texttt{=} überladen.
\begin{lstlisting}
interface operator(+)
 module procedure add
end interface operator(+)

interface assignment(=)
 module procedure gleich
end interface assignment(=)
\end{lstlisting}

Diese Operatorüberladungen brauchen immer das \texttt{intent(in)}-Attribut in den Funktionen! Überladungen von \texttt{=} brauchen dagegen sowohl \texttt{inten(in)} als auch \texttt{intent(out)}.

Mit dem \texttt{interface}-Block kann man auch lange und sperrige Namen von Funktionen einkürzen, so dass man statt \texttt{langUndSperrig(a,b)} auch \texttt{kurz(a,b)} verwenden kann:
\begin{lstlisting}
interface kurz
 module procedure langUndSperrig
end interface kurz
\end{lstlisting}